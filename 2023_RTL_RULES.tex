\documentclass[11pt]{article}
\usepackage[letterpaper, total={6.5in, 8in}]{geometry}
\usepackage{fontspec}
\usepackage[raggedright]{titlesec}
\usepackage{parskip}
\usepackage{hyperref}

\setmainfont{Atkinson Hyperlegible}

\renewcommand{\abstractname}{Summary}
\newcommand{\RTL}{Reassembly Tournament Live}
\newcommand{\termemph}[1]{\uppercase{#1}}

\begin{document}
\title{Reassembly Tournament Live Rules}
\author{Debris Field}
\date{\today}
\maketitle

\begin{abstract}
\textbf{\RTL{}} is a Tournament
series for the video game Reassembly. It's made to be fast paced and
fun to watch. Unlike most Reassembly Tournaments, where
a deadline is set well ahead of the tournament stream, in RTL,
building takes place live: two participants share their screens and
build ships under a relatively short time limit. The tournament spans
for a whole month; participants must choose a time from multiple
tournament days, allowing a greater amount of people to participate live.

\end{abstract}

\tableofcontents

\section{Logistics}
\subsection{How to Participate}
To participate in \RTL{}, please sign up for the on the Toornament page.

The form will list all days during which a \termemph{Livestream} will take place, the name you'd like to
be called on stream, your email address, your Discord username, and 
some (optional) additional info that you'd like to be mentioned during the stream.

\subsection{Dates and Times}
This tournament requires a month of weekly or bi-weekly events in order to function. On all
below dates, I will stream a few \termemph{matches}.  

\section{Tournament Structure}
After the rules are posted, participants must select all timeslots they are available for.
See ``How to Participate''.

When your entry is confirmed, I will send an email to the address specified in your entry form.
I will check to make sure that you will be available on the days you specify.
I will perform this check one day before the stream as well.

\subsection{The \termemph{Match Schedule}}
The \termemph{Match Scheudle} is the tool the tournament hosts use to keep track of
\termemph{Matches}.

The \termemph{Match Schedule} is created after the \termemph{Participation Deadline} has passed.
The \termemph{Match Schedule} is a simple double elimination bracket. The \termemph{Match Schedule}
shall be generated by Toornament. % LINK HERE!!

\section{The \termemph{Livestream}}
In \RTL{}, all shipbuilding and tournament play happen during the \termemph{Livestream}.

Participants will be prepared before the \termemph{Livestream} so that it goes smoothly.

For each round in each \termemph{match}, the tournament host selects a new set of rules. Participants are
given a Build Period to construct a rules-compliant fleet or ship. When the build period ends,
the host will fight the participants' fleets or ships together. This determines the outcome of
a round.

\subsection{\termemph{Livestream} Preparations}
Everyone who has an assigned \termemph{match} will be placed into a Discord server. 

In order to ensure that things go smoothly when the tournament goes live, before the stream, 
all participants will be placed into a Discord voice channel. I will ask all participants to open 
Reassembly and to share their screen using the website \href{https://vdo.ninja/}{VDO.Ninja}.

\subsection{During the \termemph{Livestream}}
During the \termemph{Livestream}, everyone who has an assigned \termemph{match} will be asked to stay in a voice
channel. This signifies that they are ready to be called for their \termemph{match}. 

When the \termemph{Livestream} starts, I will call \termemph{matches} in the order specified in the \termemph{match} schedule.
If someone is late, I will skip their \termemph{match}. If they do not appear
30 minutes after the final \termemph{match} of the day, their \termemph{match} will be held on the next tournament day.

When a \termemph{match} is begun, both participants will be placed into another Discord voice channel.

Both participants will be asked to share their screen using \href{https://vdo.ninja/}{VDO.Ninja}.
Both of their screens will be shown on the \termemph{Livestream}.

Participants will be given a short period to get ready. 

When both participants declare themselves as ready, the first round of the \termemph{match} shall begin.

\subsection{\termemph{match} Structure}
\termemph{matches} are always between two participants.\\
Each \termemph{match} has three (3) rounds.

Each round, the host will select a set of \termemph{Round Rules}. The participants will each be given 
one minute to read these Rules and configure their tournament settings. See the section 
``\termemph{Round Rules} Selection'' for more info.

Then, participants will be given twenty (20) minutes to construct a ship or fleet that is rules
compliant.
This time is called the \termemph{Build Period}.
When a participant is done making their ship, they must Submit it to me by posting it on the 
Discord Server in the appropriate channel. Participants may submit as many times as they want
before the end of the Build Period.

If the time limit passes before a participant has posted their file onto the Discord Server, they 
will be allowed to export their file ONCE.

After recieving a file, the host will check it against the \termemph{Round Rules}.

See ``Build Period Specifics'' for more information.

After both participants have submitted their files, the host will place them into the tournament 
page, configure the tournament settings according to the \termemph{Round Rules}, and run the
tournament. The outcome of this tournament determines who wins the round.

\section{The \termemph{Match Schedule}}


\end{document}

% will be allowed to export their file ONCE. If they are determined to be able to export the file within five (5) minutes of the 
% end of the Build Period, but do not, they will lose the match.
% If both participants in the match do not export their submission within five (5) minutes of the build time, 
